\documentclass[a4paper,11pt]{jsarticle}


% 数式
\usepackage{amsmath,amsfonts}
\usepackage{bm}
% 画像
\usepackage[dvipdfmx]{graphicx}


\begin{document}

\title{工業力学-演習問題4.12}
\author{Issaimaru}
\date{\today}
  \maketitle

  4.12 物体を同じ初速度の大きさ$v_{0}$で投げるとき,水平到達距離を最大にするためには,水平面に対してどの方向に投げればよいかを求めよ.
  \\
  \\
  (解答)\\
  まず,物体を投げてから経過した時間をt[s]とするとx方向の移動速度は,物体を投げる角度$\theta$を用いて
  \begin{equation}
    v_{x}=v_{0}\cos{\theta}
  \end{equation}
  であるから,x方向の変位はこれを0からtまでの定積分してやれば求まるので
  \begin{equation}
    x=\int_{0}^{t}v_{x}=v_{0}t\cos{\theta}
  \end{equation}
  同様に,y方向の移動速度は
  \begin{equation}
    v_{y}=v_{0}\sin{\theta}-gt
  \end{equation}
  であるから,y方向の変位は
  \begin{equation}
    y=\int_{0}^{t}v_{y}=v_{0}t\sin{\theta}-\frac{1}{2}gt^2
  \end{equation}
  と求めることができる.\\
  ここで,地面に到達するときのy方向の変位は0になるはずであるから,
  \begin{equation}
    v_{0}t\sin{\theta}-\frac{1}{2}gt^2=0
  \end{equation}
  となるときのtが投げた物体が地面に落ちるまでの時間である(t=0は投げる瞬間なので駄目)\\
  これを解くと,
  \begin{equation}
    t=\frac{2v_{0}\sin{\theta}}{g}
  \end{equation}
  となるから,これを(2)式のtに代入して
  \begin{equation}
    x=\frac{2v_{0}^2}{g}\sin{\theta}\cos{\theta}
  \end{equation}
  という式が得られる.\\
  求めたかったのはこのxが最大となるときの$\theta$であるが,(7)式の変数が$\sin{\theta}cos{\theta}$しかないことから,これが最大となるときの$\theta$を求めればよい.\\
  \begin{equation}
    \sin{\theta}cos{\theta}=\frac{sin{2\theta}}{2}
  \end{equation}
  であり,$\frac{sin{2\theta}}{2}$の最大値は1であるからこれが1になる$\theta$を見つけてやると$\theta=\frac{\pi}{4}$のときであるから,これが答えとなる.


\end{document}